\documentclass[15pt]{article}
\usepackage{geometry}           
\geometry{letterpaper}                   
\usepackage[parfill]{parskip}   
\usepackage{graphicx}
\usepackage{amssymb}
\usepackage{amsmath}
\usepackage{epstopdf}
\DeclareGraphicsRule{.tif}{png}{.png}{`convert #1 `dirname #1`/`basename #1 .tif`.png}

\title{MATH 323: Study Sheet}
\author{Ryan Ordille}
\date{Exam date: 17 April, 2012}
\begin{document}
\maketitle

\section{Before random variables}
\subsection{Basic axioms and theorems}
Principle axioms:
\begin{enumerate}
    \item $P(E) \geq 0$
    \item $P(S) = 1$
    \item For disjoint events $E_1, E_2$: $P(E_1 \cup E_2) = P(E_1) + P(E_2)$.
\end{enumerate}

Principle theorems:
\begin{enumerate}
    \item $P(\emptyset) = 0$
    \item $P(E^c) = 1 - P(E)$
    \item $P(E \cap F^c) = P(E) - P(E \cap F)$
    \item $E \subseteq F \Rightarrow P(E) \leq P(F)$
    \item For \emph{any} events $E,F$: $P(E \cup F) = P(E) + P(F) - P(E \cap F)$.
\end{enumerate}

If all outcomes are equally likely, i.e. $P(E_i) = P(E_j)$ for $i \neq j$, then:
\[
    P(E) = \frac{\textnormal{Number of ways $E$ can occur}}{\text{total number of possible outcomes}}
\]

\subsubsection{Conditional probability}
\textbf{Definition:}
\[
    P(B \; | \; A) = \frac{P(A \cap B)}{P(A)}
\]

\textbf{Law of Total Probability:} if $B_1, B_2, \dots, B_m$ form a partition of $S$ and $B_i \cap B_j = \emptyset$ for $i \neq j$, then:
\[
    P(A) = \sum_{i = 1}^m P(A \; | \; B_i) P(B_i)
\]

\textbf{Bayes' Theorem:} (assuming same premises as above)
\[
    P(B_k \; | \; A) = \frac{P(A \; | \; B_k)P(B_k)}{P(A)}
\]

Conditional expectation:
\[
    P(A \cap B) = P(B \; | \; A) P(A) = P(A \; | \; B) P(B)
\]
\subsubsection{Independence}
\[
    A \perp B \Leftrightarrow P(B \; | \; A) = P(B) \textnormal{ or } P(A \; | \; B) = P(A)
\]
\[
    A \perp B \Leftrightarrow P(A \cap B) = P(A) P(B)
\]

%%
\section{Random variable distributions}
\subsection{C.d.f. and p.f.}
\textbf{Cumulative distribution function:} (c.d.f.) $F_X (x) = P(X \leq x)$

\textbf{Probability function} (p.f.) $P_X (x) = P(X = x)$ for discrete r.v.s

\subsection{Named distribution p.f.s}
\textbf{Bernoulli $X \sim Bern(p)$:} where $P(X = 1) = p$ ``success" and $P(X = 0) = 1 - p$ ``failure"

\textbf{Binomial $X \sim Bin(n,p)$:} ``number of successes in $n$ trials" (for $x = 0, 1, \dots$)
\[
    P(X = x) = \binom{n}{x} p^x (1-p)^{n-x}
\]

\textbf{Poisson $X \sim Po(\lambda)$:} ``large $n$ and small $p$ binomial" where $np = \lambda$
\[
    P(X = x) = \frac{\lambda^x e^{-\lambda}}{x!}
\]

\textbf{Hypergeometric $X \sim HG(N,a,n)$:} ``number $a$ of successes in $n$ draws of $N$ objects"
\[
    P(X = x) = \frac{\binom{a}{x} \binom{N-a}{n-x}}{\binom{N}{n}}
\]

\textbf{Geometric $X \sim Geom(p)$:} ``trial number of first success" (for $x = 1, 2, \dots$)
\[
    P(X = x) = (1-p)^{x-1} p
\]

%%
\section{Expected values and variances}
\subsection{Definitions}
\textbf{Expected Value:} ``centre" of distribution, mean
\[
    E(X) = \mu = \sum_{x} x P_X (x)
\]
$E(cX) = cE(X)$, $E(X^2) = \sum x^2 P(X = x)$, $E(XY) \neq E(X)E(Y)$, $E(\sum X_i) = \sum E(X_i)$

\textbf{Variance:} ``spread" of the distribution
\[
    Var(X) = \sigma^2 = E((X - \mu_X)^2) = E(X^2) - \mu^2
\]
$Var(cX) = c^2 Var(X)$

\textbf{Standard deviation:} $\sqrt{Var(X)} = \sqrt{\sigma^2} = \sigma$

\subsection{Named distribution expected values and variances}
\textbf{Binomial $X \sim Bin(n,p)$:} $E(X) = np$, $Var(X) = np(1-p)$

\textbf{Bernoulli $X \sim Bernoulli(p)$:} equivalent to $Binom(1,p)$

\textbf{Poisson $X \sim Po(\lambda)$:} $E(X) = Var(X) = \lambda$

\textbf{Geometric $X \sim Geom(p)$:} $E(X) = \frac{1}{p}$, $Var(X) = \frac{1-p}{p^2}$

%%
\section{Probability density function p.d.f}
\subsection{Definitions}
As $P(X = x) = 0$ for all $x$ in a continuous distribution, our definition of the probability function is not enough for continuous distributions.
\[
    F_X (x) = P(X \leq x) = \int_{-\infty}^{x} f_X (y) \; dy
\]
\[
    E(g(X)) = \int_{-\infty}^{\infty} g(x) \; f_X (x) \; dx
\]
\[
    Var(X) = \int_{-\infty}^{\infty} x^2 \; f_X (x) \; dx - \left(\int_{-\infty}^{\infty} x \; f_X (x) \; dx \right)^2
\]
\subsection{P.d.f.s of named distributions}
\subsubsection{Uniform $X \sim U(a,b)$:}
\[
	f_X (x) = \int_a^x \frac{1}{b-a} \; dx   
\] for $a \leq x \leq b$, and $f_X (x) = 0$ elsewhere.
The c.d.f grows linearly, with $F_X (x) = \frac{x - a}{b-a}$ where $a \leq x \leq b$.

\subsubsection{Exponential $X \sim Exp(\beta)$:}
\[
    f_X (x) = \frac{1}{\beta} e^{-x / \beta} = \lambda e^{\lambda x}
\] for $x \geq 0$, with $f_X (x) = 0$ elsewhere.

\[
    F_X (x) = 1 - e^{-x / \beta}
\]

$E(X) = \mu = \beta$, $Var(X) = \sigma^2 = \beta^2$.

\textbf{Memoryless property:} if $X \sim Exp(\beta)$, then $P(x \leq X < x+h \; | \; X \geq x) = P(0 \leq x < h)$.

\subsubsection{Gamma $X \sim Gamma(\alpha, \beta)$:}
\textbf{Gamma function:} let $\alpha > 0$:
\[
    \Gamma (\alpha) = \int_0^{\infty} x^{\alpha - 1} \; e^{-x} \; dx
\] where $\Gamma(\frac{1}{2}) = \sqrt{\pi}$ and $\Gamma(\alpha + 1) = \alpha \Gamma(\alpha)$.

\textbf{Gamma distribution:}
\[
    f_X (x) = \frac{1}{\Gamma (\alpha)} \frac{1}{\beta^{\alpha}} x^{\alpha - 1} e^{-x / \beta} \text{ for } x \geq 0
\] with $f_X (x) = 0$ elsewhere.

$E(X) = \alpha \beta$, $Var(X) = \alpha \beta^2$.
\[
    F_X (x) = \int_0^x \frac{1}{\Gamma (\alpha) \beta^{\alpha}} y^{\alpha - 1} \; dy \text{ for } x > 0
\]

\subsubsection{Chi-Square $X \sim \chi_{\nu}^2$:}
\[
    f_X (x) = \frac{1}{\Gamma (\nu / 2) 2^{\nu / 2}} x^{\nu / 2 - 1} \text{ for } x \geq 0
\] with $f_X (x) = 0$ elsewhere.

\subsubsection{Gaussian/Normal $X \sim N(\mu, \sigma^2)$:}
\[
    f_X (x) = \frac{1}{\sqrt{2 \pi} \sigma} e^{-\frac{1}{2} \left(\frac{x - \mu}{\sigma}\right)^2} \; \forall x
\]
$E(X) = \mu$ and $Var(X) = \sigma^2$.

\subsection{Standardizing distributions}
\subsubsection{Any r.v.:}
To standardize any r.v. $X$ with mean $\mu$ and standard deviation $\sigma$:
\[
    Y = \frac{X - \mu}{\sigma}
\]
where $E(Y) = 0$ and $Var(Y) = 1$. 

\subsubsection{For $X \sim N(\mu, \sigma^2)$:}
\[
    Z = \frac{X - \mu}{\sigma} \sim N(0,1)
\]
where $Z \sim N(0,1)$ is a \emph{standard normal} random variable.

\section{Moment generating functions and transformations}
\subsection{Moment generating function m.g.f.}
\[
    M_X (t) = E(e^{tx}) = \int_{-\infty}^{\infty}  e^{tx} \; f_X (x) \; dx = \sum_x e^{tx} \; P(X = x)
\]
$E(X^k) = M_X^{(k)} (0)$

\subsection{Transformations}
\subsubsection{Transforming p.d.f.s}
Let $y = g(x)$ be either strictly decreasing or strictly increasing, $X$ be continuous with p.d.f. $f_X$, and $Y = g(X)$. Then, the p.d.f. of $Y$ is:
\[
    f_Y (y) = f_X (g^{-1} (y)) \left|\frac{dx}{dy}\right|
\]

\subsubsection{Probability integral transformation}
Let $X$ be a continuous random variable with a strictly increasing cumulative distribution function $F_X$, and $Y = F_X (X)$. Then, $Y \sim U(0,1)$.

\section{Joint distributions}
\subsection{Definitions}
\textbf{Joint c.d.f:} $F_{X,Y} (x,y) = P(X \leq x \cap Y \leq y) = P(X \leq x, Y \leq y)$

\textbf{Marginal c.d.f.:} 
\[
	F_X = \lim_{y \to + \infty} F_{X,Y} (x,y)
\]

\textbf{Joint p.d.f.:} (continuous)
\[
    F_{X,Y} (x,y) = \int_{-\infty}^y \int_{-\infty}^x f_{X,Y} (u,v) \; du \; dv
\]

(discrete)
\[
    F_{X,Y} (x,y) = \sum_{v \leq y} \sum_{u \leq x} P_{X,Y} (u,v)
\]

Use the Fundamental Theorem of Calculus to extract $f_{X,Y} (x,y)$.

\textbf{Joint probability function:} $P_{X,Y} (x,y) = P(X=x, Y=y)$

\textbf{Marginal p.d.f.:} integrate out the variable you wish to get rid of:
\[
    f_X (x) = \int_{-\infty}^{\infty} f_{X,Y} (x,y) \; dy
\]
\[
    f_Y (y) = \int_{-\infty}^{\infty} f_{X,Y} (x,y) \; dx
\]

\subsection{Conditional distributions}
\subsubsection{Conditional probability function}
\textbf{Definition:} given the joint probability function of $(X,Y)$, $P_{X,Y} (x,y) = P(X = x, Y = y)$:
\[
    P(Y = y \; | \; X = x) = \frac{P_{X,Y} (x,y)}{P_X (x)} = F_{Y \; | \; X \leq x} (y \; | \; X \leq x)
\]

\subsubsection{Conditional p.d.f.}
\[
    f_{Y \; | \; X = x} (y \; | \; x) = \frac{f_{X,Y} (x,y)}{f_X (x)}
\]
\[
    F_{Y \; | \; X = x} (y \; | \; x) = \int_{-\infty}^y f_{Y \; | \; X =x} (u \; | \; x) \; du
\]
\[
    E(Y \; | \; X = x) = \int_{-\infty}^{\infty} y \; f_{Y \; | \; X =x} (y \; | \; x) \; dy   
\]

\subsubsection{Law of Total Probability for Random Variables}
Discrete:
\[
    P(Y = y) = \sum_{x} P(Y = y \; | \; X = x) P(X = x) = \sum_x P_{Y \; | \; X = x} P_X (x)
\]
Continuous:
\[
    f_Y (y) = \int_{-\infty}^{\infty} f_{Y \; | \; X = x} (y \; | \; x) \; f_X (x) \; dx
\]

\section{Covariance and correlation}
\subsection{Covariance}
Continuous:
\[
    E(g(X,Y)) = \int_{-\infty}^{\infty} \int_{-\infty}^{\infty} g(x,y) f_{X,Y} (x,y) \; dx \; dy
\]
Discrete:
\[
    E(g(X,Y)) = \sum_y \sum_x g(x,y) P_{X,Y} (x,y)
\]

\textbf{Covariance:} when $g(X,Y) = (X - \mu_X) (Y - \mu_Y)$, the \emph{covariance} between $X$ and $Y$ is $E(g(X,Y))$, denoted $Cov(X,Y)$ or $\sigma_{X,Y}$.

The covariance is a measure about the means of $X$ and $Y$ -- if $Cov(X,Y) > 0$, then as $X$ increases, $Y$ also increases (same for decreasing), while if $Cov(X,Y) < 0$, then as $X$ increases, $Y$ decreases (and vice versa).

\[
    Cov(X,Y) = E(XY) - E(X)E(Y)
\]

$Cov(X,Y)$'s magnitude depends on the scale of measurement (i.e. $Cov(aX, bY) \neq Cov(X,Y)$).

\subsection{Correlation}
\textbf{Correlation:} 
\[
    |Corr(X,Y)| = |\rho(X,Y)| = \left|\frac{\sigma_{XY}}{\sigma_X \sigma_Y}\right| = \left|\frac{Cov(aX,bY)}{\sqrt{Var(aX) Var(bY)}}\right| = \frac{|ab| Cov(X,Y)}{|ab| \sqrt{Var(X) Var(Y)}}
\]
$Corr(X,Y)$ has the same sign as $Cov(X,Y)$. The correlation between $X$ and $Y$ is a measure of the linear dependence between them, \emph{not} the general dependence between them.

\subsection{Covariance and correlation}
\[
    Var(X \pm Y) = Var(X) + Var(Y) \pm 2 Cov(X,Y)
\]
If $Cov(X,Y) = 0$, then $Var(X + Y) = Var(X) + Var(Y)$, and, in general:
\[
    Var\left(\sum_{i=1}^n X_i\right) = \sum_{i=1}^n Var(X_i)
\]
if $X_i, X_j$ are uncorrelated for $i \neq j$.

\section{Independence between random variables}
The random variables $X_1, X_2, \dots, X_n$ are said to be independent if and only if
\[
    F_{X_1, X_2, \dots, X_n} (x_1, x_2, \dots, x_n) = F_{X_1} (x_1) F_{X_2} (x_2) \dots F_{X_n} (x_n)
\]

If $X$ and $Y$ are independent, then $Cov(X,Y) = 0$ and $E(XY) = E(X) E(Y)$. The converse is not always true, however. 

\section{Central Limit Theorem}
Let $X_1, X_2, \dots, X_n$ be independent and identically distributed with mean $\mu$ and variance $\sigma^2$. Then,
\[
    P\left(\left(\frac{S_n - n \mu}{\sqrt{n} \sigma}\right) \leq x\right) \rightarrow P(Z \leq x) \; \forall x \text{ as } n \rightarrow \infty
\]
\end{document}